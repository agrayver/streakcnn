%%%%%%%%%%%%%%%%%%%%%%% file template.tex %%%%%%%%%%%%%%%%%%%%%%%%%
%
% This is a general template file for the LaTeX package SVJour3
% for Springer journals.          Springer Heidelberg 2010/09/16
%
% Copy it to a new file with a new name and use it as the basis
% for your article. Delete % signs as needed.
%
% This template includes a few options for different layouts and
% content for various journals. Please consult a previous issue of
% your journal as needed.
%
%%%%%%%%%%%%%%%%%%%%%%%%%%%%%%%%%%%%%%%%%%%%%%%%%%%%%%%%%%%%%%%%%%%

\RequirePackage{fix-cm}
%
\documentclass{svjour3}                     % onecolumn (standard format)
%\documentclass[smallcondensed]{svjour3}     % onecolumn (ditto)
%\documentclass[smallextended]{svjour3}       % onecolumn (second format)
%\documentclass[twocolumn]{svjour3}          % twocolumn
%
\smartqed  % flush right qed marks, e.g. at end of proof
%
\usepackage{graphicx}

\begin{document}

\title{Particle streak velocimetry using Convolutional Neural Networks}
\titlerunning{Streaks imaging with CNN}        % if too long for running head

\author{Alexander V. Grayver         \and
        Jerome Noir
}

\institute{A. V. Grayver \at
              Institute of Geophysics, ETH Zurich \\
              Sonneggstrasse 5 \\
							8092 Zurich, Switzerland \\
              Tel.: +41-44-6333154\\
              \email{agrayver@erdw.ethz.ch}
              \and
           J. Noir \at
					    Institute of Geophysics, ETH Zurich \\
              Sonneggstrasse 5 \\
							8092 Zurich, Switzerland \\
              Tel.: +41-44-6337593\\
              \email{jerome.noir@erdw.ethz.ch}
}

\date{Received: - / Accepted: -}


\maketitle

\begin{abstract}
Insert your abstract here. 
\keywords{Streak analysis \and Neural Networks \and Turbulent Flow}
\end{abstract}

\section{Introduction}
\label{intro}

Particle Image Velocimetry (PIV) is arguably the most widely used technique to study experimental flows quantitatively \cite{raffel2018particle}. A common wokflow would consist of seeding luminescent particles and taking pairs of pictures, which register an instantaneous flow state. By splitting a pair of images into (possibly overlapping) windows and using cross-correlation between the frames, one infers the direction and magnitude. With modern computers and digital cameras, PIV has experienced wide adoption in the community \cite{JN}. For cross-correlation to work properly, one needs to ensure that the exposure time is short enough such that particles do not move more than few pixels at most. Violating this condition, for instance because of insufficient laser intensity or too fast flow, results in so called 

\section{Methods}
\label{sec:methodology}

\begin{equation}
a^2+b^2=c^2
\end{equation}

%\begin{figure*}
%\includegraphics[width=\textwidth]{example.eps}
%\caption{Please write your figure caption here}
%\label{fig:fig1}
%\end{figure*}

\section{Results}

\subsection{Network accuracy}

\subsection{Validation with DNS}

\subsection{Validation with experimental images}

\section{Conclusions}

The Juoyter Python notebooks of all programs used in this study can be found on the github \textit{Give link}.

\begin{acknowledgements}
Thank PyTorch/mathplotlib/etc authors, Meredith for DNS, Adrian for discussions.
\end{acknowledgements}


% References
%\bibliographystyle{spbasic}      % basic style, author-year citations
\bibliographystyle{spmpsci}      % mathematics and physical sciences
%\bibliographystyle{spphys}       % APS-like style for physics
\bibliography{refs}

% Non-BibTeX users please use
%\begin{thebibliography}{}

%\end{thebibliography}

\end{document}

